% Design & Architektur
% @author Tristan Ropers
%
\chapter{Design \& Architektur}

Zur Umsetzung der Anforderungen wurde eine RPC-Architektur entwickelt, die es ermöglicht den Prozess
abzubilden. Im folgenden Kapitel wird diese Architektur vorgestellt sowie ihre Besonderheiten.

\section{Systemkontext}

\includenamedimage[0.9]{../diagrams/1_systemkontext.png} {Systemkontext}

Der Systemkontext des Gesamtsystems umfasst den Nutzer, welcher gleichzeitig Administrator ist und das
Gesamtsystem. Der Nutzer konfiguriert das System und startet im Anschluss die zuvor
konfigurierte Konfiguration.
\clearpage

\section{Lösungsstrategie}

Das System besteht aus einer RPC-Architektur und einer Robotersimulation. Die RPC-Architektur ermöglicht
die Kommunikation und Steuerung der Abläufe der Roboter untereinander und dient damit als Basis der
Architektur.

\subsection{Bestimmung eines Zyklus}

Da die Nodes im System unabhängig von einer zentralen Einheit (bis auf Erhebung von Experimentdaten)
in einem Peer-To-Peer \citep{tanenbaumvansteen} Verbund existieren, muss zur Bestimmtung 
eines Zyklus eine Node die Rolle eines Koordinators \citep{tanenbaumvansteen} übernehmen, welcher den Zyklus bestimmmt.

\subsection{Schweißvorgang (Welding)}

\subsection{Logging der Prozessabläufe}

\section{Gesamtsystem}



Eine zentrale Rolle in der Architektur spielt der Roboter, er beinhaltet Kommunikationselemente (Stubs 
\citep{tanenbaumvansteen} zu den anderen Robotern), eine Statemachine sowie eine Simulation des
Schweißvorgangs.

\section{Ebenen}
