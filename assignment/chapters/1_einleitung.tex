% Anforderungsanalyse
% @author Tristan Ropers
%
\chapter{Einleitung}

In dieser Ausarbeitung geht es um die Umsetzung eines Algorithmus zum gegenseitigen Ausschlusses in einem
verteilten System von Schweißrobotern. Das Problem des wechselseitigen Ausschlusses 
ist ein weit verbreitetes Problem der Informatik. In diesem konkreten Fall sollen mehrere Schweißroboter
auf eine Ressource zugreifen können und dort einen Schweißpunkt setzen, was eine Regelung der Reihenfolge
erfordert, da die Roboter nicht alle gleichzeitig auf die gegebene Ressource zugreifen können. Um diese Reihenfolge
festzulegen wird Lamports Algorithmus \citep{lamport} zum gegenseitigen Ausschluss in verteilten Systemen umgesetzt
und für diesen Anwendungsfall evaluiert.

\section{Anforderungsanalyse}

\subsection{Requirements}

In der Aufgabenstellung sind Anforderungen an die RPC-Architektur formuliert.
Zunächst werden diese analysiert und hier zusammengefasst.
Die Anforderungen umfassen die Transparenzziele, Skalierung \cite{tanenbaumvansteen}) sowie Anforderungen
an die Umsetzung und Architektur.\\
Es wurde ein hoher Grad an Transparenz gefordert.
Die Transparenzziele umfassen die Access-Transparency, Location-Transparency, Relocation-Transparency,
Migration-Transparency, Replication-Transparency, Concurrency-Transparency und Failure-Transparency \citep{tanenbaumvansteen}.
Die Skalierung beschränkt sich administrativ auf einen Administrator, der auch gleichzeitig
Benutzer des Systems ist sowie geographisch auf einen Rechner, auf dem alle Nodes 
\citep{tanenbaumvansteen} über das Loopback-Interface der Netzwerkkarte miteinander kommunizieren.
Somit entfällt die Concurrency-Transparency, da nur ein Nutzer am System beteiligt ist sowie die
Location-Transparency, Relocation- und Migration-Transparency, da alle Nodes über das Loopback-Interface
kommunizieren, welches auf eine Netzwerkkarte beschränkt ist. Die Replication-Transparency ist in diesem Zusammenhang 
uninteressant, da jede Node im System eindeutig identifizierbar ist,was Replikationen der Nodes für die Funktionsweise 
der verwendeten Algorithmen ausschließt.\\\\
Desweiteren ist ein Minimalset an Funktionen an die RPC-Architektur gefordert:

\begin{itemize}
 \item void register(int id)
 \item void welding()
 \item void setStatus(int status)
\end{itemize}

Es soll kein zentrales System, bis auf Erfassung von Daten zu Experimentabläufen, am Gesamtsystem beteiligt
sein.
Alle beteiligten Nodes (Roboter) sollen sich auf einen Zyklus einigen, in dem immer drei Roboter nacheinander
(Reihenfolge durch Lamport) schweißen.
Insgesamt soll kein Roboter mehr als drei Schweißpuntke mehr als ein anderer gesetzt haben.
Alle Roboter sollen am Ende eines Experiemntablaufs mindestens 20 Schweißpunkte gesetzt haben.
In 99\% soll ein Roboter nach einem Schweißvorgang weiterhin betriebsbereit sein. Im Umkehrschluss
ist ein Roboter in 1\% der Fälle nach einem Schweißvorgang nicht mehr betriebsbereit.\\
Zur Auswertung und Beobachtung des Ablaufs soll jeder Roboter seinen Ablauf loggen.
Daraus ergeben sich folgende Anforderungen:

\begin{table}[h!]
\begin{center}
\begin{tabular}{ |p{2.5cm}|p{11cm}| } 
 \hline
 Requirement & Registrierung eines Nodes im System \\
 \hline
 Beschreibung & Ein Node kann sich im System mit allen anderen Nodes bekannt machen. \\
 \hline
 Eingaben & id: int; eindeutige ID für den Node \\
 \hline
 Ziel & Die Node hat sich mit allen anderen im System bekanntgemacht. \\
 \hline
 Vorbedingung & Die Node ist hochgefahren und betriebsbereit. \\
 \hline
 Nachbedingung & Die Node kennt alle anderen Nodes im System und alle anderen Nodes kennen die sich 
 registrierende Node. \\
 \hline
\end{tabular}
\caption{Requirement Registrierung}
\label{table:reqregister}
\end{center}
\end{table}

\begin{table}[h!]
\begin{center}
\begin{tabular}{ |p{2.5cm}|p{11cm}| } 
 \hline
 Requirement & Ein Roboter versucht zu schweißen \\
 \hline
 Beschreibung & Ein Roboter einer Node möchte Zugriff auf die Ressource haben und einen Schweißauftrag
 ausführen. Sobald die Ressource verfügbar ist, soll der Roboter seinen Schweißauftrag ausführen.\\
 \hline
 Eingaben & / \\
 \hline
 Ziel & Der Roboter hat einen Schweißvorgang durchgeführt. \\
 \hline
 Vorbedingung & Die Node sowie der Roboter der Node ist hochgefahren und betriebsbereit. \\
 \hline
 Nachbedingung & Der Roboter hat einen Schweißvorgang abgeschlossen und geht entweder in einen Fehlerzustand
 (in 1\% der Fälle) oder bleibt betriebsbereit. \\
 \hline
\end{tabular}
\caption{Requirement Schweißen (Welding)}
\label{table:reqwelding}
\end{center}
\end{table}

\begin{table}[h!]
\begin{center}
\begin{tabular}{ |p{2.5cm}|p{11cm}| } 
 \hline
 Requirement & Bestimmung eines Zyklus \\
 \hline
 Beschreibung & Es muss sichergestellt sein, dass immer genau drei Roboter pro Zyklus einen
 Schweißauftrag ausführen. \\
 \hline
 Eingaben & / \\
 \hline
 Ziel & Ein Zyklus wird bestimmt und drei Roboter können schweißen. \\
 \hline
 Vorbedingung & Alle am Experiment teilnehmenden Nodes sowie dazugehörige Roboter sind
 hochgefahren und betriebsbereit. \\
 \hline
 Nachbedingung & Ein Zyklus wurde bestimmt und drei Nodes haben den Schweißauftrag erhalten. \\
 \hline
\end{tabular}
\caption{Requirement Bestimmung eines Zyklus}
\label{table:reqchoosecycle}
\end{center}
\end{table}

\begin{table}[h!]
\begin{center}
\begin{tabular}{ |p{2.5cm}|p{11cm}| } 
 \hline
 Requirement & Logging der Prozessabläufe \\
 \hline
 Beschreibung & Jede Node muss seinen Zustand und die Abläufe loggen. \\
 \hline
 Eingaben & logText: String; Event oder Zustand \\
 \hline
 Ziel & Ein Event oder Zustand des Nodes wurde geloggt. \\
 \hline
 Vorbedingung & Die Node sowie der Roboter der Node ist hochgefahren und betriebsbereit. \\
 \hline
 Nachbedingung & Der Zustand der Node oder das Event wurden erfolgreich geloggt. \\
 \hline
\end{tabular}
\caption{Requirement Logging}
\label{table:reqlogging}
\end{center}
\end{table}

\begin{table}[h!]
\begin{center}
\begin{tabular}{ |p{2.5cm}|p{11cm}| } 
 \hline
 Requirement & Austauschbarkeit des Algorithmus \\
 \hline
 Beschreibung & Der Algorithmus für den gegenseitigen verteilten Ausschluss soll austauschbar als Bibliothek
 in die Architektur eingebunden werden.\\
 \hline
\end{tabular}
\caption{Requirement Austauschbarkeit des Algorithmus}
\label{table:reqmutexexchangable}
\end{center}
\end{table}

\begin{table}[h!]
\begin{center}
\begin{tabular}{ |p{2.5cm}|p{11cm}| } 
 \hline
 Requirement & Randbedingungen des Prozessablaufs \\
 \hline
 Beschreibung &
 \begin{itemize}
  \item Nach dem Ablauf des Experiments sollen alle Roboter mindestens 20 Schweißpunkte
  gesetzt haben oder sich in einem Fehlerzustand befinden.
  \item Kein Roboter darf mehr als drei Schweißpunkte als ein anderer gesetzt haben.
  \item Die Anzahl der teilnehmenden Nodes (Roboter) liegt zwischen drei und 16.
 \end{itemize} \\
 \hline
\end{tabular}
\caption{Requirement Prozessbedingungen}
\label{table:reqprocess}
\end{center}
\end{table}

\clearpage

\subsection{UseCase}
\begin{table}[h!]
\begin{center}
\begin{tabular}{ |p{2.8cm}|p{11cm}| } 
 \hline
 UseCase & Prozessablauf mit mindestens drei Robotern \\
 \hline
 ID & UC01 \\
 \hline
 Beschreibung & Dieser UseCase schildert einen Prozessablauf von mindestens drei Robotern, wie
 er im Experiment durchgeführt wurde. \\
 \hline
 Vorbedingungen & Mindestens drei Roboter und der Logger werden hochgefahren.\\
 \hline
 Hauptszenario & 
 \begin{enumerate}
  \item Alle Roboter registrieren sich gegenseitig.
  \item Ein Zyklus wird bestimmt.
  \item Alle Roboter, welche im Zyklus ausgewählt wurden,\newline schweißen nacheinander.
  \item Ein neuer Zyklus wird bestimmt.
  \item Wiederholen ab Punkt 2.
 \end{enumerate} \\
 \hline
 Alternative\newline Szenarien &
 A1: \newline
 \begin{enumerate}
  \setcounter{enumi}{3}
  \item Mindestens ein Roboter ist in den Fehlerzustand gewechselt.
  \item Versuch neuen Zyklus zu bestimmen und zu schweißen.
 \end{enumerate}  \\
 \hline
 Fehlerszenarien & 
 E1: \newline
 \begin{enumerate}
   \setcounter{enumi}{2}
   \item Schweißaufträge im Zyklus können nicht abgeschlossen werden.
   \item System geht in Fehlerzustand.
 \end{enumerate} \\
 \hline
 Nachbedingung & Alle Roboter haben mindestens 20 Schweißpunkte gesetzt oder sind in einem Fehlerzustand. \\
 \hline
 Ergebnis & Das Experiment wurde erfolgreich beendet und alles wurde geloggt. \\ 
 \hline
\end{tabular}
\caption{UseCase Prozessablauf mit mindestens drei Robotern}
\label{table:usecase1}
\end{center}
\end{table}
