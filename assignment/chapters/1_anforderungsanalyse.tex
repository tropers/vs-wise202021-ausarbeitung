% Anforderungsanalyse
% @author Tristan Ropers
%
\chapter{Anforderungsanalyse}

In der Aufgabenstellung sind Anforderungen an die RPC-Architektur formuliert.
Zunächst werden diese analysiert und hier zusammengefasst.
Die Anforderungen umfassen die Transparenzziele, Skalierung \cite{tanenbaumvansteen}) sowie Anforderungen
an die Umsetzung und Architektur.\\
Es wurde ein hoher Grad an Transparenz gefordert.
Die Transparenzziele umfassen die Access-Transparency, Location-Transparency, Relocation-Transparency,
Migration-Transparency, Replication-Transparency, Concurrency-Transparency und Failure-Transparency \citep{tanenbaumvansteen}.
Die Skalierung beschränkt sich administrativ auf einen Administrator, der auch gleichzeitig
Benutzer des Systems ist sowie geographisch auf einen Rechner, auf dem alle Nodes 
\citep{tanenbaumvansteen} über das Loopback-Interface der Netzwerkkarte miteinander kommunizieren.
Somit entfällt die Concurrency-Transparency, da nur ein Nutzer am System beteiligt ist sowie die
Location-Transparency, Relocation- und Migration-Transparency, da alle Nodes über das Loopback-Interface
kommunizieren, welches auf eine Netzwerkkarte beschränkt ist. Die Replication-Transparency ist in diesem Zusammenhang 
uninteressant, da jede Node im System eindeutig identifizierbar ist,was Replikationen der Nodes für die Funktionsweise 
der verwendeten Algorithmen ausschließt.\\\\
Desweiteren ist ein Minimalset an Funktionen an die RPC-Architektur gefordert:
\begin{itemize}
  \item void register(int id)
  \item void welding()
  \item void setStatus(int status)
\end{itemize}
\newpage
Es soll kein zentrales System, bis auf Erfassung von Daten zu Experimentabläufen, am Gesamtsystem beteiligt
sein.
Alle beteiligten Nodes (Roboter) sollen sich auf einen Zyklus einigen, in dem immer drei Roboter nacheinander
(Reihenfolge durch Lamport) schweißen.
Insgesamt soll kein Roboter mehr als drei Schweißpuntke mehr als ein anderer gesetzt haben.
Alle Roboter sollen am Ende eines Experiemntablaufs mindestens 20 Schweißpunkte gesetzt haben.
In 99\% soll ein Roboter nach einem Schweißvorgang weiterhin betriebsbereit sein. Im Umkehrschluss
ist ein Roboter in 1\% der Fälle nach einem Schweißvorgang nicht mehr betriebsbereit.\\
Daraus ergeben sich folgende Anforderungen:

\begin{center}
Registrierung \\
\begin{tabular}{ |p{2.5cm}|p{11cm}| } 
 \hline
 Requirement & Registrierung eines Nodes im System \\
 \hline
 Beschreibung & Ein Node kann sich im System mit allen anderen Nodes bekannt machen. \\
 \hline
 Eingaben & id: int; eindeutige ID für den Node \\
 \hline
 Ziel & Die Node hat sich mit allen anderen im System bekanntgemacht. \\
 \hline
 Vorbedingung & Die Node ist hochgefahren und betriebsbereit. \\
 \hline
 Nachbedingung & Die Node kennt alle anderen Nodes im System und alle anderen Nodes kennen die sich 
 registrierende Node. \\
 \hline
\end{tabular}
\end{center}

\begin{center}
Schweißen (Welding) \\
\begin{tabular}{ |p{2.5cm}|p{11cm}| } 
 \hline
 Requirement & Ein Roboter versucht zu schweißen \\
 \hline
 Beschreibung & Ein Roboter einer Node möchte Zugriff auf die Ressource haben und einen Schweißauftrag
 ausführen. \\
 \hline
 Eingaben & / \\
 \hline
 Ziel & Der Roboter hat einen Schweißvorgang durchgeführt. \\
 \hline
 Vorbedingung & Die Node sowie der Roboter der Node ist hochgefahren und betriebsbereit. \\
 \hline
 Nachbedingung & Der Roboter hat einen Schweißvorgang abgeschlossen. \\
 \hline
\end{tabular}
\end{center}
