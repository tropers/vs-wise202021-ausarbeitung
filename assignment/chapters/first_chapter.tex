% first example chapter
% @author Tristan Ropers
%
\chapter{Anforderungsanalyse}

In der Aufgabenstellung sind Anforderungen an die RPC-Architektur formuliert.
Zunächst werden diese analysiert und hier zusammengefasst.
Die Anforderungen lassen sich in die nichtfunktionalen (Transparenzziele, Skalierung \cite{tanenbaumvansteen}) und funktionale (Architektur, Design und Software-Features) unterteilen.\\
Bei den nichtfunktionalen Anforderungen wurde ein hoher Grad an Transparenz gefordert.
Die Transparenzziele umfassen die Access-Transparency, Location-Transparency, Relocation-Transparency,
Migration-Transparency, Replication-Transparency, Concurrency-Transparency und Failure-Transparency \citep{tanenbaumvansteen}.\\
Die Skalierung beschränkt sich administrativ auf einen Administrator, der auch gleichzeitig
Benutzer des Systems ist sowie geographisch auf einen Rechner, auf dem alle Nodes 
\citep{tanenbaumvansteen} über das Loopback-Interface der Netzwerkkarte miteinander kommunizieren.
Somit entfällt die Concurrency-Transparency, da nur ein Nutzer am System beteiligt ist sowie die
Location-Transparency, Relocation- und Migration-Transparency, da alle Nodes auf einem Rechner
als eigenständige Prozesse laufen. Die Replication-Transparency ist in diesem Zusammenhang uninteressant,
da jede Node im System eindeutig identifizierbar ist, was Replikationen der Nodes für die
Funktionsweise der verwendeten Algorithmen ausschließt.
